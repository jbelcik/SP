\documentclass[12pt,a4paper]{article}

\usepackage[left=1.6in,right=.8in,top=1.5in,bottom=1.5in]{geometry}
\usepackage{polski}
\usepackage[utf8]{inputenc}
\usepackage[pdftex,linkbordercolor={0 0.9 1}]{hyperref}
\usepackage{amsthm,amsmath,amsfonts,amssymb,mathrsfs}
\usepackage{url}
\usepackage{enumerate}
\usepackage{graphicx}

\newtheorem{tw}{Twierdzenie}[section]
\newtheorem{stw}[tw]{Stwierdzenie}
\newtheorem{fakt}[tw]{Fakt}
\newtheorem{lemat}[tw]{Lemat}

\theoremstyle{definition}
\newtheorem{df}[tw]{Definicja}
\newtheorem{ex}[tw]{Przyk³ad}
\newtheorem{uw}[tw]{Uwaga}
\newtheorem{wn}[tw]{Wniosek}
\newtheorem{zad}{Zadanie}

\DeclareMathOperator{\R}{\mathbb{R}}
\DeclareMathOperator{\Z}{\mathbb{Z}}
\DeclareMathOperator{\N}{\mathbb{N}}
\DeclareMathOperator{\Q}{\mathbb{Q}}

\providecommand{\abs}[1]{\left\lvert#1\right\rvert}
\providecommand{\var}[1]{\operatorname{var}(#1)}

\usepackage{fancyhdr}
\pagestyle{fancy}
\fancyhf{}
\fancyfoot[R]{\textbf{\thepage}}
\fancyhead[L]{\small\sffamily \nouppercase{\leftmark}}
\renewcommand{\headrulewidth}{0.4pt}
\renewcommand{\footrulewidth}{0.4pt}

\title{Funkcje ci±g³e i róczkowalne}
\date{\today}

\author{Jakub Be³cik}


\begin{document}



\maketitle

\tableofcontents

\section{Funkcje ci±g³e}

\begin{df}
(funkcja ci±g³a). Niech $f: (a,b) \rightarrow \mathbb{R}$, oraz niech $x_0 \in (a,b)$. Móy, ¿e instrukcja $f$ jest ci±g³a w punkcie $x_0$ wtedy i tylko wtedy, gdy:
\[ \forall_{\epsilon >0}\exists_{\delta >0}\forall x \in (a,b) |x-x_0| < \delta  \Rightarrow  |f(x) - f(x_0)| < \epsilon .\]
\end{df}

\begin{ex}
Wielomiany, funkcje trygonometryczne, wyk³adnicze, logarytmiczne s± ci±g³e w ka¿dym punkcie swojej dziedziny.
\end{ex}

\begin{ex}
Funkcja $f$ dana wzorem:
\[ f (x) = \left\{ \begin{array}{ll} x + 1 & \mbox{dla } x \neq  0 \\ 0 & \mbox{dla } x = 0 \end{array} \right. \]
Jest ci±g³a w ka¿ym punkcie poza $x_0 = 0$.
\\ Niech $\mathbb{Q}$ oznacza zbiószystkich liczb wymiernych.
\end{ex}

\begin{ex}
Funkcja $f$ dana wzorem:
\[ f (x) = \left\{ \begin{array}{ll} 0 & \mbox{dla } x \in  \mathbb{Q} \\ 1 & \mbox{dla } x \notin  \mathbb{Q} \end{array} \right. \]
Nie jest ci±g³a w ¿adnym punkcie.
\end{ex}

\newpage

\begin{ex}
Funkcja $f$ dana wzorem:
\[ f (x) = \left\{ \begin{array}{ll} 0 & \mbox{dla } x \in  \mathbb{Q} \\ x & \mbox{dla } x \notin  \mathbb{Q} \end{array} \right. \]
Jest ci±g³a w punkcie $x_0 = 0$, ale nie jest ci±g³a w pozosta³ych punktach dziedziny.
\end{ex}

\begin{zad}
Udowodnij prawdziwo¶æodanych przyk³adó\end{zad}

\begin{df}
Je¶li funkcja $f: A \rightarrow  \mathbb{R}$ jest ci±g³a w ka¿dym punkcie swojej dziedziny $A$ to móy kró, ¿e jest ci±g³a.
\[ \mbox{Poni¿sze twierdzenie zbiera podstawowe w³asno¶ci zbioru funkcji ci±g³ych.} \]
\end{df}

\begin{tw}
Niech funkcje $f, g: R \rightarrow  \mathbb{R}$ bê ci±g³e, oraz niech $\alpha , \beta  \in  \mathbb{R}$. Wtedy funkcje:
\begin{enumerate}[a)]
\item $h_1(x) = \alpha  \cdot  f(x) + \beta  \cdot  g(x)$,
\item $h_2(x) = f(x) \cdot  g(x)$,
\item $h_3(x) = \frac{f(x)}{g(x)}$ (o ile $g(x) \ne  0$ dla dowolnego $x \in \mathbb{R})$,
\item $h_4(x) = f(g(x))$,
\end{enumerate}
S± ci±g³e.
\end{tw}

Nie chcia³o mi siêego przepisywaæo wklei³em jaki¶ tekst o bobrach z google.pl... We wczesnym ¶redniowieczu bóeuropejski (Castor fiber L.) zamieszkiwa³ licznie ca³± Europê Azjêd strefy stepóo tundrêJednak w pocz±tkach wieku XX przetrwa³o jedynie osiem ma³ych populacji gromadz±cych w sumie ok. 1200 osobnikóGatunek stan±³ w obliczu gro¼by wyginiêa. W wiêzo¶ci krajóniknê wraz z wiekiem XIX.

\begin{tw}
Niech $f: [a,b] \rightarrow  \mathbb{R}$ ci±g³a, oraz niech $f(a) \ne  f(b)$. Wtedy dla dowolnego $y_0 \in $ conv$\{f(a),f(b)\}$ istnieje $x_0 \in  [a,b]$ takie, ¿e $f(x_0) = y_0$.
\end{tw}

\section{Róczkowalno¶æ
\begin{df}
Niech $F: (a,b) \rightarrow  \mathbb{R}, x_0 \in  (a,b)$ oraz $f$ ci±g³a w otoczeniu punktu $x_0$. Je¶li istnieje granica:
\[ \lim_{x\rightarrow x_0} \frac{f(x) - f(x_0)}{x - x_0} \]
I jest skoñna, to oznaczamy j± przez $f'(x_0)$ i nazywamy pochodn± funkcji $f$ w punkcie $x_0$.
\end{df}

\begin{df}
Je¶li funkcja $f$ posiada pochodn± w ka¿dym punkcie swojej dziedziny, to móy, ¿e $f$ jest róczkowalna. Istnieje wtedy funkcja $f'$, któka¿demu punktowi z dziedziny funkcji $f$ przyporz±dkowuje warto¶æochodnej pochodnej funkcji $f$ w tym punkcie.
\end{df}

\begin{ex}
Wielomiany, funkcje trygonometryczne, wyk³adnicze, logarytmiczne s± róczkowalne w ka¿dym punkcie dziedziny.
\end{ex}

\begin{ex}
Funkcja $f(x) = |x|$ jest ci±g³a, ale nie posiada pochodnej w punkcie $x_0 = 0$.
\end{ex}

\begin{tw}
Niech $f: [a,b] \rightarrow  \mathbb{R}$ ci±g³a i róczkowalna na (a,b). Dodatkowo niech $f'(x) \ne  0$ dla $x \in  (a,b)$, oraz niech $m = min_{x\in [a,b]}f(x), M = max_{x\in [a,b]}f(x)$. Wtedy na pewno $f(a) = m, f(b) = M$ lub $f(a) = M$ i $f(b) = m$.
\end{tw}



\end{document}
